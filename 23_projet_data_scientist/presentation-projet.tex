\documentclass{book}

\usepackage[utf8]{inputenc}
\usepackage[T1]{fontenc}
\usepackage[francais]{babel}
\usepackage{graphicx} 
\usepackage{fancyref}
\usepackage{hyperref}
\usepackage{url}


\title{%
  Projet de Sciences des Données \\
  \large Explotation d'images satellites pour la prevision de croissance économique \\
    }

\author{\textsc{Youcef} - \textsc{Kacer}}
\date{25 Aout 2016}

\begin{document}
 
\maketitle

\tableofcontents

\frontmatter
\chapter{Introduction}
Ce document présente le projet de sciences des données que je compte développer, dans le cadre de la validation du CES Data Scientist à Telecom-Paristech \cite{cesds}.
Il consiste à exploiter des images satellitaires haute-résolution afin d'en extraire des indicateurs socio-économiques.\\
En effet, évaluer la densité démographique d'un pays, par exemple, peut représenter un co\^{u}t non négligeable en terme de 
recensement, or utiliser des images aériennes et leurs caractéristiques permettrait de prédire la population présente à moindre co\^{u}t.
En effet, les \og edges \fg des routes et des batiments caractérisent les zones urbaines et donc les zones à forte population, alors que
les champs et les for\^{e}ts caractérisent des zones faiblement peuplées.\\

\mainmatter
\chapter{Images exploitées}

Les images à exploiter proviennent du satellite Landsat 8 de la NASA et sont libres d'accès \cite{landsat8}. Ce satellite scanne tout le globe terrestre 
tous les 16 jours. Elles permettent donc non seulement d'étudier une zone à un moment donnée mais aussi d'étudier son évolution sur
une période donné.\\
Ces images sont très riches car elles présentent en tout 11 canaux, 9 dans le visible et 2 dans l'infra-rouge, pour des résolutions 
allant de 15 à 60 mètres. Donc en plus des caractéristiques de formes, le niveau des images doit pouvoir renseigner sur 
les matérieux présents au sein d'un zone (métal ou végétation par exemple).\\

\chapter{Méthode d'apprentissage}

L'idée serait de s'intéresser à une certaine zone (un pays par exemple), dont on aurait l'indicateur de densité de population (valeur 
à prédire) en fonction de la commune du pays.\\
On pourra alors récupérer plusieurs images satellitaires quadrillant cette zone, et attribuer à chacune d'elles
sa valeur de densité de population (on doit pouvoir utiliser la latitude et la longitude d'une image pour retrouver la commune concernée).\\
Ainsi, on récupère un ensemble classique d'images labelisées.\\
Ensuite, on pourra extraire des descripteurs de ces images (histogramme orienté du gradient \cite{Dalal05histogramsof}, entre autre)
auquels on appliquera un algorithme de regression supervisé (la valeur à prédire, la densité de
polpulation, est plut\^{o}t continue que discrète).\\
On pourra alors tester la généralisation du classifieur, en s'intéressant à d'autres pays.\\

\chapter{Outils}

Les images peuvent 

\clearpage

\backmatter

\listoftables

\listoffigures

\bibliographystyle{alpha}
\bibliography{biblio}

\end{document}